\chapter{Conclusion and Future work}
\label{chp:conclusion}

In this thesis, we proposed to perform {\em selective} knowledge transfer for CF problems and came up with a systematical study on how the factors such as variance of empirical errors could leverage the selection.
We found although empirical error is effective to model the consistency across domains, it would suffer from the sparseness problem in CF settings. By introducing a novel factor - variance of empirical errors to measure how trustful this consistency is, the proposed criterion can better identify the useful source domains and the helpful proportions of each source domain.
We embedded this criterion into a boosting framework to transfer the most useful information from the source
domains to the target domain.
The experimental results on real-world data sets showed that our selective transfer learning solution performs better than several state-of-the-art methods at various sparsity levels.
Furthermore, comparing to existing methods, our solution works well on long-tail users and is more robust to overfitting.

However, we notice that there are limitations in the work. First, in STLCF, the knowledge transfer is item-based. That is, each item / user is evaluated independently. Therefore, the implicit relationships between items / users are omitted. Second, The computational cost of STLCF is expensive, even though the parallel implementation makes it possible to run on large clusters. Third, we require the full correspondence on either user set or the item set as a bridge for the knowledge transfer. This requirement limits the applications of STLCF in the real world, because most of the real system will not be able to provide the full correspondence information. Fourth, we are aware that although the STLCF performs well on the long-tails (target domain tasks with very limited observations, for example the experiment in Section \ref{sec:long-tail}), it still can not handle the case where no record of target domains is exist.

We believe the Selective Transfer Learning has practical applications in the real world and would be a promising research topic. STLCF is our initial attempt on this topic. In the future to make Selective Transfer Learning be more robust, we propose the following approaches:
\begin{itemize}
\item {\bf Model-based Selective Transfer Learning.} Instead of item-based knowledge transfer, we would like to explore the model-based transfer. That is, the domain information is first generalized as model and then be applied to later tasks. This will improve the universality of the source domain information and reduce the storage demand, as the information is generalized by models.
\item {\bf Relationship Regularized Selective Transfer Learning.} On the one hand, with the rapid growing of social networks in the internet, we have access to plenty of online user relationship. On the other hand, previous researches on taxonomy have made it possible to build relationship between items. Relationship regularized selection of helpful knowledge is naturally the next work.
\item {\bf Selection of Domain Correspondence.} Due to either record corruption or the absence of data in the industry, it is not always possible to obtain the full correspondence between the source domains and the target domain for knowledge transfer. To make Selective Transfer Learning be practical, we want to research on the selection of correspondence between domains when only part of it could be helpful. For example, in the settings where the user set is shared among the source and the target domains, we would like to select parts of the users during the transfer learning processes.
\item {\bf Boosting in Multi-Dimension.} The technique in this article can be viewed as boosting over either the item or user dimension. Can we extend it to multi-dimensional boosting? For example, would the interest of a user towards certain items evolute over time? With the evolution of user interests, is it possible to make a better prediction on the future ratings?
\end{itemize} 