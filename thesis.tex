%%%%%%%%%%%%%%%%%%%%%%%%%%%%%%%%%%%%%%%%%%%%%%%%%%%%%%%%%%%%%%%%%%%%%%%%%
%                                                                       %
% ustthesis_test.tex: A template file for usage with ustthesis.cls      %
%                                                                       %
%%%%%%%%%%%%%%%%%%%%%%%%%%%%%%%%%%%%%%%%%%%%%%%%%%%%%%%%%%%%%%%%%%%%%%%%%

\documentclass{ustthesis}

% \usepackage{latexsym}
    % Use the "latexsym" package when encountering the following error:
    %   ! LaTeX Error: Command \??? not provided in base LaTeX2e.
% \usepackage{epsf}
    % Use the "epsf" package for including EPS files.

%%%%%%%%%%%%%%%%%%%%%%%%%%%%%%%%%%%%%%%%%%%%%%%%%%%%%%%%%%%%%%%%%%%%%%%%%
%                                                                       %
% Preambles. DO NOT ERASE THEM. Change to suite your particular purpose.%
%                                                                       %
%%%%%%%%%%%%%%%%%%%%%%%%%%%%%%%%%%%%%%%%%%%%%%%%%%%%%%%%%%%%%%%%%%%%%%%%%

\usepackage{enumitem}
\usepackage{times} %
\usepackage{helvet} %
\usepackage{courier} %
\usepackage{multirow} %
\usepackage{epsfig} %
\usepackage{verbatim} %
\usepackage{xcolor}
\usepackage{amsmath}
\usepackage{amssymb}
\usepackage{graphicx}
\usepackage[linesnumbered,ruled,vlined]{algorithm2e}
\usepackage{subfigure}
\usepackage{epsfig}
\usepackage{listings}
\usepackage{float}
\usepackage[a4paper,top=2.5cm,bottom=2.5cm,left=2.5cm,right=2.5cm]{geometry}
\usepackage{cases}
\usepackage{url}
\usepackage{algorithmic}
\lstset{ %
  backgroundcolor=\color{white},   % choose the background color; you must add \usepackage{color} or \usepackage{xcolor}
  basicstyle=\footnotesize,        % the size of the fonts that are used for the code
  breakatwhitespace=false,         % sets if automatic breaks should only happen at whitespace
  breaklines=true,                 % sets automatic line breaking
  captionpos=b,                    % sets the caption-position to bottom
  %commentstyle=\color{mygreen},    % comment style
  deletekeywords={...},            % if you want to delete keywords from the given language
  escapeinside={\%*}{*)},          % if you want to add LaTeX within your code
  extendedchars=true,              % lets you use non-ASCII characters; for 8-bits encodings only, does not work with UTF-8
  frame=single,                    % adds a frame around the code
  keepspaces=true,                 % keeps spaces in text, useful for keeping indentation of code (possibly needs columns=flexible)
  keywordstyle=\color{blue},       % keyword style
  language=Octave,                 % the language of the code
  morekeywords={*,...},            % if you want to add more keywords to the set
  %numbers=left,                    % where to put the line-numbers; possible values are (none, left, right)
  %numbersep=5pt,                   % how far the line-numbers are from the code
  %numberstyle=\tiny\color{mygray}, % the style that is used for the line-numbers
  rulecolor=\color{black},         % if not set, the frame-color may be changed on line-breaks within not-black text (e.g. comments (green here))
  showspaces=false,                % show spaces everywhere adding particular underscores; it overrides 'showstringspaces'
  showstringspaces=false,          % underline spaces within strings only
  showtabs=false,                  % show tabs within strings adding particular underscores
  %stepnumber=2,                    % the step between two line-numbers. If it's 1, each line will be numbered
  %stringstyle=\color{mymauve},     % string literal style
  tabsize=2,                       % sets default tabsize to 2 spaces
  title=\lstname                   % show the filename of files included with \lstinputlisting; also try caption instead of title
}

\def\etal{{\em et al.\/}\,}
\newcommand{\Us}{\mathcal{U}}
\newcommand{\Is}{\mathcal{V}}
\newcommand{\Loss}{\mathcal{L}}
\newcommand{\R}{\mathcal{R}}
\newcommand{\X}{\mathbf{X}}
\newcommand{\W}{\mathbf{w}}



\title{Transfer Learning for One-Class Recommendation Based On Matrix Factorization}  % Title of the thesis.
\author{Ruiming Xie}     % Author of the thesis.
\degree{\MPhil}             % Degree for which the thesis is.
%% or
%\degree{\PhD}              % Degree for which the thesis is.
\subject{Computer Science and Engineering}      % Subject of the Degree.
\department{Computer Science and Engineering}       % Department to which the thesis
                    % is submitted.
\advisor{Prof. Qiang Yang}     % Supervisor.
\depthead{Prof. Qiang Yang}    % department head.
%\acting\depthead{Prof.~Siu-Wing~Cheng}    % department head.
\defencedate{2014}{12}{18}      % \defencedate{year}{month}{day}.

% NOTE:
%   According to the sample shown in the guidelines, page number is
%   placed below the bottom margin.  However, if the author prefers
%   the page number to be printed above the bottom margin, please
%   activate the following command.

% \PNumberAboveBottomMargin



\newtheorem{example}{\textbf{Example}}[chapter]
\numberwithin{definition}{chapter}

\begin{document}

%%%%%%%%%%%%%%%%%%%%%%%%%%%%%%%%%%%%%%%%%%%%%%%%%%%%%%%%%%%%%%%%%%%%%%%%%
%                                                                       %
% Now the actual Thesis. The order of output MUST be followed:          %
%                                                                       %
%    1) TITLEPAGE                                                       %
%                                                                       %
% The \maketitle command generates the Title page as well as the        %
% Signature page.                                                       %
%                                                                       %
%%%%%%%%%%%%%%%%%%%%%%%%%%%%%%%%%%%%%%%%%%%%%%%%%%%%%%%%%%%%%%%%%%%%%%%%%

\maketitle

%%%%%%%%%%%%%%%%%%%%%%%%%%%%%%%%%%%%%%%%%%%%%%%%%%%%%%%%%%%%%%%%%%%%%%%%%
%                                                                       %
%     2) DEDICATION (Optional)                                          %
%                                                                       %
% The \dedication and \enddedication commands are optional. If          %
% specified it generates a page for dedication.                         %
%
%%%%%%%%%%%%%%%%%%%%%%%%%%%%%%%%%%%%%%%%%%%%%%%%%%%%%%%%%%%%%%%%%%%%%%%%%

% \dedication
% This is an optional section.
% \enddedication

%%%%%%%%%%%%%%%%%%%%%%%%%%%%%%%%%%%%%%%%%%%%%%%%%%%%%%%%%%%%%%%%%%%%%%%%%
%                                                                       %
%     3) ACKNOWLEDGMENTS                                                %
%                                                                       %
% \acknowledgments and \endacknowledgments defines the                  %
% Acknowledgments of the author of the Thesis.                          %
%                                                                       %
%%%%%%%%%%%%%%%%%%%%%%%%%%%%%%%%%%%%%%%%%%%%%%%%%%%%%%%%%%%%%%%%%%%%%%%%%
\acknowledgments
The past two years and a half research would have been much harder without the guidance of my supervisor, assistance from my friends and support of my parents.

Firstly, I would like to express my deepest gratitude to my supervisor Prof. Qiang Yang for his instruction on both my research and my life. Without his guidance, I would never have the chance to exploit so many possibilities, in particular my interest in recommender system.

I would also like to express my greatest appreciation to Lili Zhao and Liya Ji for their companionship and valuable suggestions on my study, also for their kindness and tolerance as my roommates.

In addition, I would like to thank Ben Tan, Bo Liu, Bin Wu, Kaixiang Mo, Zhongqi Lv, Lianghao Li for their ideas, suggestion and company. Also thanks to Dr. Wei Xiang and Dr. Qian Xu for their guidance on my future career. I would like to thank Dr. Yong Li, Xiaoping Lai, Xiaopeng Zhang, Lei Xiao for their encouragement and criticism.

I am grateful to Prof. Dit-Yan Yeung and Prof. Chi-Wing Wong for serving on my thesis examination committee. Many thanks to the CSE staffs, Mr. Isaac Ma and Ms. Connie Lau, for their administration work.

Finally, I am thankful to my mother for her support and belief in me.

I wish all my friends good luck, and have fun!




%%%%%%%%%%%%%%%%%%%%%%%%%%%%%%%%%%%%%%%%%%%%%%%%%%%%%%%%%%%%%%%%%%%%%%%%%
%                                                                       %
%     4) TABLE OF CONTENTS                                              %
%                                                                       %
%%%%%%%%%%%%%%%%%%%%%%%%%%%%%%%%%%%%%%%%%%%%%%%%%%%%%%%%%%%%%%%%%%%%%%%%%

\tableofcontents

%%%%%%%%%%%%%%%%%%%%%%%%%%%%%%%%%%%%%%%%%%%%%%%%%%%%%%%%%%%%%%%%%%%%%%%%%
%                                                                       %
%     5) LIST OF FIGURES (If Any)                                       %
%                                                                       %
%%%%%%%%%%%%%%%%%%%%%%%%%%%%%%%%%%%%%%%%%%%%%%%%%%%%%%%%%%%%%%%%%%%%%%%%%

\listoffigures

%%%%%%%%%%%%%%%%%%%%%%%%%%%%%%%%%%%%%%%%%%%%%%%%%%%%%%%%%%%%%%%%%%%%%%%%%
%                                                                       %
%     6) LIST OF TABLES (If Any)
%                                                                       %
%%%%%%%%%%%%%%%%%%%%%%%%%%%%%%%%%%%%%%%%%%%%%%%%%%%%%%%%%%%%%%%%%%%%%%%%%

\listoftables

%%%%%%%%%%%%%%%%%%%%%%%%%%%%%%%%%%%%%%%%%%%%%%%%%%%%%%%%%%%%%%%%%%%%%%%%%
%                                                                       %
%     7) ABSTRACT                                                       %
%                                                                       %
% \abstract and \endabstract are used to define a short Abstract for    %
% the Thesis.                                                           %
%                                                                       %
%%%%%%%%%%%%%%%%%%%%%%%%%%%%%%%%%%%%%%%%%%%%%%%%%%%%%%%%%%%%%%%%%%%%%%%%%

\abstract
{\bf C}ollaborative {\bf F}iltering (CF) aims to predict users' ratings on items according to historical user-item preference data. In many real-world applications, preference data are usually sparse, which would make models overfit and fail to give accurate predictions.
Recently, several research works show that by transferring knowledge from some manually selected source domains, the data sparseness problem could be mitigated.
However for most cases, parts of the source domain data are {\em not consistent} with the observations in the target domain, which may misguide the target domain model building.
In this paper, we propose a novel criterion based on empirical prediction error and its variance to capture the consistency across domains in CF settings. Consequently, we embed this criterion into a boosting framework to perform {\em selective} knowledge transfer.
Comparing with several state-of-the-art methods, we show that our proposed selective transfer learning framework can significantly improve the accuracy of rating prediction on several real-world recommendation tasks.


%%%%%%%%%%%%%%%%%%%%%%%%%%%%%%%%%%%%%%%%%%%%%%%%%%%%%%%%%%%%%%%%%%%%%%%%%
%                                                                       %
%     8) The Actual Contents                                            %
%                                                                       %
% The command \chapters MUST BE USED to ensure that the entire content  %
% of the Thesis is double-spaced (in version 1.0).                      %
%                                                                       %
% However, in version 2.0, \chapters will be automatically added in     %
% the beginning of the first chapter.                                   %
%                                                                       %
%%%%%%%%%%%%%%%%%%%%%%%%%%%%%%%%%%%%%%%%%%%%%%%%%%%%%%%%%%%%%%%%%%%%%%%%%

%%\chapters         % Not necessary with ustthesis.cls (v2.0).

%%%%%%%%%%%%%%%%%%%%%%%%%%%%%%%%%%%%%%%%%%%%%%%%%%%%%%%%%%%%%%%%%%%%%%%%%
%                                                                       %
% Each chapter is defined via the \chapter command. The usual sectional %
% commands of LaTeX are also available.                                 %
%                                                                       %
%%%%%%%%%%%%%%%%%%%%%%%%%%%%%%%%%%%%%%%%%%%%%%%%%%%%%%%%%%%%%%%%%%%%%%%%%


\chapter{Introduction}
\label{chp:intro}

\hspace{0.1in}
\section{Motivation}
Recommendation systems have become extremely common in recent years, typical recommendation system recommends items (movies, music, books, etc.) that users may be interested in. Collaborative filtering approaches build a model from users' past behavior to predict items that the user may have an interest in. 
In real-world recommendation systems, users and items are all very large, so users can only rate a small fraction of items. Thus, the user-item matrix can be extremely sparse. What's more, sometimes we can't observe explicit ratings, only implicit feedback is provided(e.g click, pageview and purchase). Such problem may lead to poor performance in CF models.

Recently, different transfer learning methods have been developed to improve the performance of the model.In \cite{/ijcai/libin09, /icml/libin09}, they use a rating-pattern sharing scheme to share user-item ratings pattern across different domains. In \cite{/aaai/WPan12, Pan:2011:TLP:2283696.2283784}, implicit feedback data is available, knowledge is transferred via latent-feature sharing. In \cite{/uai/ZhangCY10, DBLP:conf/aaai/EldardiryN11} they try to exploit correlations among multiple domains.
However, most of the methods are develeoped for rating prediction problems. For example, in a music \& book rating website, a user can have high or low rating for an album. The ratings are usually trustful, thus can be used to recommend books to the same users. But in a website where only implicit feedback is available(e.g advertisement), the behavior can be much more noisy and with less information. So to acheive better performance, we much transfer more knowledge from source domain while be very careful about the noise.

Some works have been done on solving one-class recommendation problem \cite{4781121, 4781145}. They all try to model the freqency of actions by a confidence matrix. For example, if you clicked an item A for 10 times, item B for 1 time. It's more confident that you like A, but not quite sure that you like B. On the other side, if you are a heavy user and you didn't click a popular item A, then it's highly possible that you don't like A. But these works only explore the original matrix, in real-world there are many other useful informations which can be used to improve performance.

We collect several users' clicking and purchasing behaviors from two online shopping site. After taking careful analysis, we find that users' behaviors on clicking and purchasing are similar, but not the same. Based on that, we develop a matrix tri-factorization method(TRIMF) to transfer knowledge from side to side. TRIMF can be used to achieve different goals, (e.g optimize for Ctr(Cvr)).

Further, to make the method online, we develeop a clustering-based matrix factorization method(CBMF) using hadoop. CBMF collect all kinds of user data and convert them into a single matrix per task. For cold-start users, a weighted recommendation from their neighbors will be provided. While for registered users, results are mixed with direct matrix factorization and CBMF.

\hspace{0.1in}
\section{Contributions}

Our main contributions are summarized as follows:

\begin{itemize}[noitemsep,topsep=0pt,parsep=0pt,partopsep=0pt]
\item First, we find that in implicit datasets, more data must be shared to acheive better performance. To transfer more knowledge, a matrix tri-factorization method is proposed to transfer knowledge from user side and item side(TRIMF).
\item Second, implicit datasets can consist many noises. To transfer useful knowledge, we develop a clustering-pattern transfer function. For each task, a base clustering pattern matrix is provided, the function only do some cluster-level transformation. Thus we can share knowledge more accurately without losing too much information.
\item Third, we propose a modified version of TRIMF which can be used for large scale recommendation. And it is used in an Internet company, it's performance is among the best in all online algorithms.
\end{itemize}

\hspace{0.1in}
\section{Thesis Outline}

The rest of the thesis is organized as follows: we first provide the background of the research on Transfer Learning, Collaborative Filter and Matrix Factorization in Chapter \ref{chp:bg}. Then, we discuss the technique grounds of the proposed matrix tri-factorization method in Chapter \ref{chp:trimf}. We present the details of our proposed STLCF framework in Chapter \ref{chp:cbmf} . Finally, we share our thoughts of possible future work and conclude the thesis in Chapter \ref{chp:conclusion}.


%%%%%%%%%%%%%%%%%%%%%%%%%%%%%%%%%%%%%%%%%%%%%%%%%%%%%%%%%%%%%%%%%%


\chapter{Background}
\label{chp:bg}

In this chapter, we would like to give a brief review of the related literatures.
We classify our work to be most related to the works in the areas of collaborative filtering.

In Table~\ref{tbl:relatedW}, we summarize the related works under the collaborative filtering context.
To the best of our knowledge, no previous work for collaborative filtering has ever focused on the fine-grained analysis of knowledge transfer between source domains and the target domain, i.e., the selective transfer learning.

In the following, we would like to discuss the state-of-the-art methods for both Collaborative Filtering and Transfer Learning.


\begin{table}[h]
\caption{Overview of STLCF in a big picture of Collaborative Filtering.}
\begin{Large}
\label{tbl:relatedW}
\begin{center}
\begin{tabular}{ c || c | c }
\hline\hline
& Selective & Non-Selective \\
\hline\hline
\multirow{1}{*} {Transfer} & \textbf{\bf \em STLCF} & RMGM~\cite{/ijcai/libin09}, CMF~\cite{/kdd/SinghG08},\\
\multirow{1}{*} {Learning} & & TIF~\cite{/aaai/WPan12}, etc.\\
\cline{1-3}
\multirow{1}{*} {Non-Transfer} & -- & MMMF~\cite{rennie2005fast}, GPLSA~\cite{DBLP:conf/sigir/Hofmann03}, \\
\multirow{1}{*} {Learning} &  & PMF~\cite{/nips/SalakhutdinovM07}, etc.\\
\hline\hline
\end{tabular}
\end{center}
\end{Large}
\end{table}

\hspace{0.1in}
\section{Collaborative Filtering}
As an intelligent component in recommendation systems, Collaborative Filtering (CF) has gained extensive interest in both academia and industry.
Generally speaking, CF is a method of making automatic predictions (filtering) about the interests of a user by collecting preferences or taste information from many users (collaborating). The underlying assumption of the collaborative filtering approach is that, if a person A has the same opinion as B on an issue, A is more likely to have B's opinion on a different issue x than to have the opinion on x of a randomly chosen person. For example, a collaborative filtering recommendation system for television tastes could make predictions about which television show a user should like given a partial list of this user's tastes (likes or dislikes, ratings, etc).

There are three types of CF: memory-based, model-based and hybrid.

\hspace{0.05in}
\subsection{Memory-based CF}
This mechanism uses user rating data to compute the similarity between users or items. The similarity is then used for making recommendations. The memory-based method is used in many commercial systems, because it is easy to implement and is effective given plenty of records. Typical examples of this mechanism are neighborhood based CF and item-based/user-based top-N recommendations\cite{su2009survey}.

The advantages of this approach include:
\begin{itemize}
\item The explainability of the results, which is an important aspect of recommendation systems.
\item It is easy to setup and use.
\item New data can be added easily and incrementally.
\item It need not consider contents of the items being recommended.
\item The mechanism scales well with co-rated items.
\end{itemize}

However, there are several disadvantages with this approach:
\begin{itemize}
\item It requires plenty of human ratings.
\item Its performance decreases when data gets sparse, which is a common phenomenon with web related items.
\item Although it can efficiently handle new users, adding new items becomes more complicated since that representation usually relies on a specific vector space. That would require to include the new item and re-insert all the elements in the structure. This prevents the scalability of this approach.
\end{itemize}

\hspace{0.05in}
\subsection{Model-based CF}
Models are developed using data mining, machine learning algorithms to find patterns based on training data. This approach has a more holistic goal to uncover latent factors that explain observed ratings. Most of the models are based on creating a classification or clustering technique to identify the users in the test set.
Various models have been proposed, including factorization models~\cite{/computer/yehuda09matrix, /aaai/WPan12,paterek07,/tist/LibFM-TIST12},
probabilistic mixture models~\cite{hofmann04cf,jin:decoupled}, Bayesian networks~\cite{pennock00pd} and restricted Boltzman machines~\cite{/icml/SalakhutdinovMH07}.

There are several advantages with this paradigm:
\begin{itemize}
\item It handles the sparsity better than memory based ones.
\item This helps with scalability with large data sets.
\item It improves the prediction performance.
\item It gives an intuitive rationale for the recommendations.
\end{itemize}

The disadvantage of this approach is the expensive model building. On the one hand, the modern recommendation system usually have petabytes of records as input; On the other hand, the convergence of most models requires intensive computation. One needs to have a tradeoff between prediction performance and scalability.

Given the accuracy of model-based CF, how to overcome the scalability issue has attracted much concern. With the rapid development of parallel computation, researchers have been exploring the use of parallel system to speed up the complex model building. For example in ~\cite{chu2007map}, the authors showed that a variety of machine learning algorithms including k-means, logistic regression, naive Bayes, SVM, PCA, gaussian discriminant analysis, EM and backpropagation (NN) could be speeded up by Google's map-reduce ~\cite{dean2008mapreduce} paradigm.

Being aware of the computational infeasibility for most useful CF models in the real world settings, we would like to adopt the parallel computation framework in our implementation and experiments. We classified our Selective Transfer Learning for CF as model-based CF and use parallel computing to well handle the real world data and complex modeling.

\hspace{0.05in}
\subsection{Hybrid models}
A number of applications \cite{das2007google} combine the memory-based and the model-based CF algorithms. These overcome the limitations of native CF approaches. It improves the prediction performance. Importantly, it overcomes the CF problems such as sparsity and loss of information. However, they have increased complexity and are expensive to implement.

\hspace{0.1in}
\section{Transfer Learning} Pan and Yang ~\cite{/tkde/sinno09survey} surveyed the field of transfer learning. A major assumption in many machine learning and data mining algorithms is that the training and future data must be in the same feature space and have the same distribution. However, in many real-world applications, this assumption may not hold. For example, we sometimes have a classification task in one domain of interest, but we only have sufficient training data in another domain of interest, where the latter data may be in a different feature space or follow a different data distribution. In such cases, knowledge transfer, if done successfully, would greatly improve the performance of learning by avoiding much expensive data-labeling effort.

Recently, researchers propose the MultiSourceTrAdaBoost~\cite{/cvpr/YaoD10} to allow automatically selecting the appropriate data for knowledge transfer from multiple sources. The newest work TransferBoost~\cite{/aaai/Eatond11} was proposed to iteratively construct an ensemble of classifiers via re-weighting source and target instance via both individual and task-based boosting. Moreover, EBBoost~\cite{/jmlr/ShivaswamyJ10a} suggests weight the instance based on the empirical error as well as its variance.

\hspace{0.05in}
\subsection{Transfer Learning for Collaborative Filtering}
Some works on transfer learning are in the context of collaborative filtering.
Mehta and Hofmann~\cite{/ki/bhaskar06cross} consider the scenario involving two systems with shared users and use manifold alignment methods to jointly build neighborhood models for the two systems. They focus on making use of an auxiliary recommender system when only part of the users are aligned, which does not distinguish the consistency of users' preferences among the aligned users.
Li \etal~\cite{/icml/libin09} designed a regularization framework to transfer knowledge of cluster-level rating patterns, which does not make use of the correspondence between the source and the target domains.

Our work is the first to systematically study {\em selective} knowledge transfer in the settings of collaborative filtering. Besides, we propose the novel factor - variance empirical error that is shown to be of much help in solving the real world CF problems.

\hspace{0.05in}
\subsection{Negative Transfer}
Negative transfer happens when the source domain data and task contribute to the reduced performance of learning in the target domain.
This could happen in the context of recommendation system. In the traditional transfer learning framework, if two users have common interests on the light music, we tend to believe they share the similar opinions on books. However, two users who have common interests on the light music may have quite different tastes on the books. In this case, if the transfer learning techniques are applied anyway, we can expect the bad performance. We call this phenomenon negative transfer.

Despite the fact that how to avoid negative transfer is a very important issue, little research work has been published on this topic. Rosenstein et al. \cite{rosenstein2005transfer} empirically showed that if two tasks are too dissimilar, brute-force transfer may hurt the performance of the target task. Some works have been exploited to analyze relatedness among tasks and task clustering techniques, such as ~\cite{ben2003exploiting} and ~\cite{bakker2003task}, which may help provide guidance on how to avoid negative transfer automatically. Bakker and Heskes ~\cite{bakker2003task} adopted a Bayesian approach in which some of the model parameters are shared for all the tasks and others more loosely connected through a joint prior distribution that can be learned from the data. Thus, the data are clustered based on the task parameters, where tasks in the same cluster are
supposed to be related to each other. Argyriou et al. \cite{argyriou2008algorithm} considered situations in which the learning tasks can be divided into groups. Tasks within each group are related by sharing a low-dimensional representation, which differs among different groups. As a result, tasks within a group can find it easier to transfer useful knowledge.

In our work, we investigate the negative transfer issue in the context of the recommendation system scenario. The key is to identify the items which would cause either much uncertainty in knowledge transfer or reduction in performance. Subsequently, we proposed a selective transfer learning framework for the model-based collaborative filtering tasks.

\hspace{0.05in}
\subsection{Large Scale Transfer Learning}
So far, transfer learning has been mostly considered in the off-line learning settings, which do not emphasize the scalability and computation speed. Due to the rapid development of storage technique and flourish of internet services, the real world problems in recent recommendation systems are mostly based on some large data sets. Little work on large scale transfer learning has been published in previous literature, though it is badly desirable. To cope with the growing needs of today's recommendation system, we would like to investigate the parallel framework in our experiments. There are already some researchers working on the large scale machine learning, as you may find in a post from quora \footnote{\url{http://www.quora.com/What-are-some-software-libraries-for-large-scale-learning}}. In our approach, we tried the Map-Reduce Framework and the Message Passing Interface (MPI).

\hspace{0.02in}
\subsubsection{Map-Reduce Framework}
MapReduce is a framework for processing parallelizable problems in huge datasets using a large number of computers (nodes). A MapReduce program comprises a Map() procedure that performs filtering and sorting (such as sorting students by first name into queues, one queue for each name) and a Reduce() procedure that performs a summary operation (such as counting the number of students in each queue, yielding name frequencies).
\begin{itemize}
\item {\bf ``Map" step:} The master node takes the input, divides it into smaller sub-problems, and distributes them to worker nodes. A worker node may do this again in turn, leading to a multi-level tree structure. The worker node processes the smaller problem, and passes the answer back to its master node.
\item {\bf ``Reduce" step:} The master node then collects the answers to all the sub-problems and combines them in some way to form the output, i.e. the answer to the problem it was originally trying to solve.
\end{itemize}
We will show that our methods can be plugged into the Map-Reduce framework for parallelization.

\hspace{0.02in}
\subsubsection{Message Passing Interface}
Message Passing Interface (MPI) is a standardized and portable message-passing system designed by a group of researchers from academia and industry to function on a wide variety of parallel computers. Both point-to-point and collective communication are supported. As a dominant model used in high-performance computing, MPI's goals are high performance, scalability, and portability. There are several advantages of MPI:
\begin{itemize}
\item {\bf Universality.} The message-passing model fits well on separate processors connected by a either fast or slow communication network.
\item {\bf Expressivity.} MPI exposes powerful interface to express the data parallel models.
\item {\bf Ease of debugging.} Because we write shared-memory (machine learning) models under MPI, the debugging process is relatively easier.
\item {\bf Performance.} As modern CPUs have become faster, management of their caches and the memory hierarchy has become the key to getting the most out of the machines. Message passing provides a way for the programmer to explicitly associate specific data with processers and thus allow the compiler and cache-management hardware to function fully.
\end{itemize}
Due to the flexibility of the protocol provided under MPI, it works well for the code development of learning algorithms. We will use MPI for implementation.

\chapter{Transfer Learning in One Class CF for Shopping Prediction}
\label{chp:trimf}
\section{Problem settings}
\subsection{Background}
\par{
In real-world, a person usually has different kinds of behaviors before buying one thing. For online shopping sites, their goal is to let users buy their products, but the user-item matrix for deal is extremely sparse( less than $0.001\%$ ). Therefore, if we only use the available information, we cannot achieve a reliable or even reasonable performance.

In an online shopping site, there are two main actions - click and purchase. Both can form a matrix which consists of only binary data(1-action happened, 0-unknown). Let's denote $X_d$ to be the matrix of deal, $X_c$ to be the matrix of click. We know that deal matrix $X_d$ is very sparse and although click matrix $X_c$ is also sparse, it is much denser than $X_d$. To use more data, we develop a transfer learning algorithm(TRIMF) that leverages click data to predict purchasing. Compared with former methods which only share either rating patterns or latent features , our method shares both rating patterns and latent features through cluster-level transformation and overlapping matrices. Experiments in \ref{sec:offline} show that our algorithm performs better than other baseline(transfer and non-transfer) methods.}

\subsection{Problem definition}
  \begin{itemize}
  \item Input: [0-1 matrix:user click matrix $X_C(m_c*n_c)$, user deal matrix $X_d(m_d*n_d)$] , $m_c, m_d$ denote the number of users, $n_c, n_d$ denote the number of items. Users and items partially overlap.
  \item Output: Two prediction matrix $P_C(m_c*n_c), P_d(m_d*n_d)$, which predict users' purchasing behavior.
  \end{itemize}



\section{TRIMF}
\subsection{Weighting scheme of TRIMF}
  \par{Former one-class CF methods ~\cite{4781121}, ~\cite{4781145} use weighted low-rank approximation to tackle the problem that all observed ratings are 1. Given a rating matrix $R = (R_{ij})_{m*n} \in \{0, 1\}^{m*n}$ with $m$ users and $n$ items and a corresponding non-negative weight matrix $W = (W_{ij})_{m*n} \in R^{m*n^+}$ , weighted low-rank approximation aims at finding a low rank matrix $X = (X_{ij})_{m*n} $minimizing the objective of a weighted Frobenius loss function as follows : $L(X) = \|\sum W_{ij}(R_{ij} - X_{ij})\|_2$. 

In ~\cite{4781121}, the authors consider actions that happen more than once(e.g. multiple clicks on an item). Negative entries are ignored; for each positive entry, its weight is proportional to its frequency, since higher frequency can mean that we are more confident about the entry. For example, user $i$ viewed item $j$, $n$ times, then $W_{ij} = 1 + log(n)$. In ~\cite{4781145}, positive entries all have same weight 1, while negative entries are considered differently. According to their experiments, the user-oriented weighting scheme can achieve the best performance. That is, for negative entries $W_{ij} \propto \sum_j{R_{ij}}$, the idea is that if a user has more positive examples, it is more likely that the user does not like the other items, that is, the missing data for this user is negative with higher probability.

In our method, we adopt these weighting schemes to give missing values proper weights, that is, for positive entries we use the weighting scheme in ~\cite{4781121} and for negative entries we use user-oriented weighting. $$ W_{ij}=\left\{
\begin{aligned}
1 + log(n) & & X_{ij} = 1\\
log(\sum_j{R_{ij}}) &  & X_{ij} = 0 \\
\end{aligned}
\right.
$$}
\subsection{Transfer learning in TRIMF}

Usually users' deal data is very sparse. For instance, users will buy $n_d$ items in one day while clicking $n_c$ items which often results in $n_d \ll n_c$. Therefore, only using deal data is not sufficient. Traditional transfer learning methods use matrix factorization and share a certain part of low-rank matrices to achieve knowledge transfer(e.g. user-latent factor, rating pattern). However, none of them apply the selective-sharing scheme as ours does.

In TRIMF, rating matrices are factorized into three parts : $X = USV^T$. The first part $U$ stands for user clustering results or latent factor, $V$ stands for item clustering results or the latent factor, while $S$ stands for the clustering relationships between user clusters and item clusters. We want to learn a better cluster-level rating pattern $S$ from users' deal data with the help of users' click data, not just use users' deal data. Therefore we factorize two matrices $X_c, X_d$ together. In order to transfer knowledge, we must make sure that their latent spaces are the same. For a user who has click and deal actions, it would be particularly beneficial that his latent vectors factorized from $X_c, X_d$ are the same.

Therefore, for overlap users and items, we want their clustering vector  $U,V$ to be the same. What is more, we want even more knowledge transfer from the matrix $S$ which stands for cluster relationship or rating patterns. However, what a user likes to click is not always the item he wants to buy. In Yixun, there are only two common items in the top 10 click items and top 10 purchase items (Table ~\ref{tbl:topitem}). Therefore these rating patterns should somehow be related but not the same. We cannot simply make the pattern matrix $S$ the same in the prediction. 

\begin{table}[h]

%\begin{Large}
\label{tbl:topitem}
\begin{center}
\begin{tabular}{| c | c |}
\hline
Top click items & Top purchase items \\
\hline
Iphone 5s & Tissue\\
Xiaomi 3 & Laundry soap powder\\
Thinkpad & Xiaomi 3\\
CPU & Snacks\\
Hard disk & Battery\\
Router & Iphone 5s\\
Earphone & Mouse\\
\hline
\end{tabular}
\caption{Top 10 click items and purchase items in Yixun.}
\end{center}
%\end{Large}
\end{table}

\par{After careful observation we found that there are some items which belong to the same category with a higher conversion rate (user tends to buy after clicking), but not with other categories. There are also some users who like window-shopping while others buy an item straight after clicking. These are all cluster-level features. We design a mapping function to allow the learnt $S$ better suit the data.

We design two mapping vectors: $U,V$, if we have learnt a rating pattern $S_c$ from a click matrix, then for the deal matrix pattern we have $S_d^{ij} = U_i * S_c^{ij} * V_j$. The transformation is based on $S_c$ to enable knowledge transfer, while after being multiplied by $U$ and $V$ we can capture the difference between them, at the cluster-level. }

\subsection{Object function}
\par{We use a weighted non-negative matrix tri-factorization method to deal with the problem as illustrated below. 
\begin{figure}

%\begin{Large}

\begin{center}
\includegraphics[width=400px]{fig/trimf.jpg} 
\caption{Graphical model of TRIMF.}
\label{fig:trimf}
\end{center}
\end{figure}}
 
  \par{Objective Function:$$min_{F,G,S,U,V} W_c\odot ||X_c - (F;F_c)S(G;G_c)'||_2 + W_d\odot ||X_d - (F;F_d)(USV)(G;G_d)'||_2 $$}

  \par{
    \begin{itemize}
    \item $W_c,W_d$ are the weights for $X_C, X_d$, every observed entry has weight $1 + log(frequency)$. While others have weight $W_{ij} = \sum_jI({R_{ij}})$.
    \item $F, G$ are the soft clustering result matrices for overlapped users(items), they are forced to be the same. $F_c,F_d,G_c,G_d$ are matrices for unique users(items).
    \item $U,V$ are two diagonal matrices, $U_{ii}$ scales every $S_{i*}$ to $U_{ii}S_{i*}$, and models the users' cluster-level transformation from click to deal. While $V_{jj}$ scales every $S_{*j}$ to $S_{*j}V_{jj}$, it models the items' cluster-level transformation from click to deal.
    \item When predicting, we use $(F;F_d)(USV)(G;G_d)$ to predict users who have deal data. Then since we got two mapping matrices $U,V$, we apply $U,V$ back to the click pattern matrix $S$ to predict for users who have click data, i.e. we use $(F;F_c)(USV)(G;G_c)$.
    \end{itemize}
}



\begin{section}
  {Solution to TRIMF \& Algorithm}
\par{Following the update rules in ~\cite{Zhuang:2011:EAW:1952191.1952195}, we use an alternately iterative algorithm to solve the objective function.}

Firstly, we declare some denotations:
\begin{itemize}
	\item $Ic,Icc,Icd,Id$ : $(Ic,Icc)*(F;F_c) = I*(F;F_c)$ and $(Icd,Id)*(F;F_d) = I*(F;F_d)$
	\item $sg$ : $S*G'*G*S'$
	\item $F_1, F_2$ : $[F;F_c]$, $[F;F_d]$
\end{itemize}
\par{In each round of iterations these matrices are updated as :
$$F \leftarrow F .* \sqrt{\frac{Icc'*(W_c.*X_c)*G*S' + Icd'*(W_d.*X_d)*G*S'}{(Icc'*Icc*F + Icc'*Ic*F_c + Icd'*(Icd*F + Id*F_d))*sg)}}$$
$$F_c \leftarrow F_c .* \sqrt{\frac{Ic'*(W_c.*X_c)*G*S'}{Ic'*(Icc*F + Ic*F_c)*sg}}$$
$$F_d \leftarrow F_d .* \sqrt{\frac{Id'*(W_d.*X_d)*G*S'}{Id'*(Icd*F + Id*F_d)*sg }}$$
$$G \leftarrow G .* \sqrt{\frac{W_c.*X_c*F_1*S + (W_d.*X_d)'*F_2*S}{(G*(S'*F_1'*F_1*S + S'*F_2'*F_2*S)}}$$
$$U \leftarrow U .* \sqrt{\frac{F_2'*(W_d.*X_d)*G*V'*S'}{F_2'*F_2*U*S*V*G'*G*V'*S'}}$$
$$V \leftarrow V .* \sqrt{\frac{S'*F_2'*(W_d.*X_d)*G}{S'*F_2'*F_2*S*V*G'*G}} $$



}

\par{The user-item matrix is typically very sparse with $z \ll nm$ non-zero entries while $k$ is also much smaller than n, m. Through using sparse matrix multiplications and avoiding dense intermediate matrices, the update steps can be very efficiently and easily implemented. In particular, updating F, S, G each takes $O(k^2 (m + n) + kz)$ , and the algorithm usually reaches convergence in less than 200 iterations.}

\begin{algorithm}[tb]
\caption{Algorithm for TRIMF.}
\begin{algorithmic}

\STATE {\bfseries Input:} $\X_{c}$, $\X_{d}$\\
$\X_{c} \in \mathbb{R}^{m_c\times n_c}$: the purchase data \\
$\X_{d} \in \mathbb{R}^{m_d\times n_d}$: the click data\\

\STATE {\bfseries Initialize:} Initialize $W_c, W_d$ : $(1+log(freq))$ for observed, $\sum_jI({R_{ij}})$ for unseen, $F,G,S,U,V$ : $random$, Set overlap numbers for users and items

\FOR{ $i$ = 1 to $T$}

\STATE update $F$

\STATE update $F_c$, $F_d$

\STATE  update $G$

\STATE  update $S$

\STATE  update $U, V$


\ENDFOR

\STATE {\bfseries Output:} $F,G,S,U,V$

\end{algorithmic}
\label{algorithm:TRIMF}
\end{algorithm}

\end{section}

\chapter{Clustering-based Matrix Factorization for Online Shopping Prediction}
\label{chp:cbmf}
\section{Limitation of TRIMF}
In Chapter \ref{chp:trimf}, we introduced TRIMF. It's a matrix tri-factorization method for cross-domain recommendation. However, it has some limitations which restricts its scalability and extensibility.

First, when data are coming from multiple sources(e.g click, pageview, add cart), TRIMF treats every source equally and put each of them into a matrix which is very sparse. When solving the object function, more matrix will increase the time complexity and space complexity. If we try to update $S$, every matrix is included so it will be very time-consuming.

What's more, in reality we can't ignore users with fewer actions. Thus the matrix will be much more sparse than the ones in our experiment, so we can't guarantee to acheive equal performance.

To solve these problems, we developed a framework based on clustering and scoring scheme (CBMF). CBMF firstly cluster users according to their behaviour and demongraphic features, then automatically convert different types of actions into one matrix call action matrix, finally a matrix factorization method is applied in the action matrix. For users with enough actions, personalized recommendation is provided. Otherwise we provide a recommendation based on his/her cluster.

\section{Clustering method of CBMF}

Usually users' actions are unique and sparse, it'll be time-consuming if we want to cluster users using raw data. In Tencent, we have 800,000,000 users in total, and their feature vector size can be as large as 1,000,000. So if we want to speed up the phase, we must first convert large sparse user vector into a low dimension dense vector.

\subsection{Simhash}

Simhash is a kind of locally sensitive hashing(LSH). LSH is a hashing method that if we got two points $A,B$ which are close in their original space, after hashing we got $A',B'$, then $A',B'$ is still close in the new space. Thus we keep the relationship of distance among two spaces. 

The input of Simhash per user is $(feature_1, weight_1),..(feature_n,weight_n)$. The procedure of Simhash is in Algorithm \ref{algorithm:simhash}.

\begin{algorithm}[tb]
\caption{Simhash Algorithm for one instance.}
\begin{algorithmic}

\STATE {\bfseries Input:} $\X_U$, $h$\\
$\X_{U}$ : $(feature_1, weight_1),..(feature_n,weight_n)$ \\
$h$: a smooth hash function with $k$ bits after hashing\\

\STATE {\bfseries Initialize:} $r$(result vector) : $[0,0,0..0] \in \{0,1\}^k$

\FOR{ $i$ = 1 to $n$}

\STATE calculate $h(feature_i)$

\STATE $r = r + weight_i * h(feature_i)$

\ENDFOR

\FOR{ $i$ = 1 to $k$}
\STATE {if $r_i > 0$}
\STATE {$r_i = 1$}
\STATE {else $r_i = 0$}

\ENDFOR

\STATE {\bfseries Output:} $r$

\end{algorithmic}
\label{algorithm:simhash}
\end{algorithm}

Assume that Simhash convert a vector $x$ into a 32-dimension binary vector $x'$. Actually $i_{th}$ bit of $x'$ is the sign of inner product of $x$ and $H_i = [h^1_i, h^2_i,...h^n_i]$, $H_i$ can be regarded as a pingmian in original space. If two vector $x, y$ are in the same direction of $H_i$, then $x', y'$ is equal on $i_{th}$ bit. Thus we can use hamming distance in the low dimension to represent their similarity in original space.

\subsection{Minhash}


\input{content/exp}
\chapter{Conclusion and Future work}
\label{chp:conclusion}

In this thesis, we proposed to perform knowledge transfer for one-class CF problems using matrix factorization and came up with a matrix tri-factorization method and an online framework to systematically study on the factors that will affect selection.
We found although there exist some methods which tackle the one-class CF problem and there are also some transfer learning methods for CF. But no one has deal with one-class CF in multiple domains. Simply apply former transfer learning methods will fail due to data sparsity. 
We found under matrix tri-factorization framework(TRIMF), we can transfer as much knowledge as we can while ignore the noise. 
By leveraging overlapped users and items, we can transfer knowledge from different domains. While applying the linear control factor to pattern matrix, we can avoid direct transfer which can bring noise while capture the similarity between different domains.
To put our method in reality, we developed a clustering based matrix factorization framework(CBMF) which automatically integrate all data together then perform matrix factorization.
The experimental results for TRIMF in real-world data sets showed that our method performs better than several state-of-the-art methods in conversion rate comparison.
The experimental results for CBMF in real-world showed that our method has the best conversion rate and moderate click-through rate among others.
 
However, we notice that there are limitations in the work. First, in TRIMF, the computational cost is expensive since multiplicative rules will affect all matrix in update time. Second, we only support non-negative matrix factorization in TRIMF, because we need to constrain non-negative to fulfill optimization conditions. If matrix can be negative, it'll be more flexible and can carry more information. Third, both TRIMF and CBMF are point-wise methods which optimize for each entry of the matrix, actually we only need to rank those items not to calculate their score. That is, we only need their relative relationship \cite{Rendle:2009:BBP:1795114.1795167}. Fourth, TRIMF and CBMF are batch updated algorithms, but in online test almost all algorithms whose performance are good are real-time.


We believe that Transfer Learning for One-Class Recommendation has practical applications in the real world and would be a promising research topic. TRIMF/CBMF represents our initial attempt on this topic. In the future to make it more robust, we propose the following approaches:
\begin{itemize}
\item {\bf Pair-wise Transfer Learning in CF.} Instead of point-wise transfer in CF, pair-wise CF is becoming more and more popular because it can almost achieve better results. In \cite{DBLP:dblp_conf/recsys/LercheJ14, DBLP:dblp_conf/recsys/Aiolli14} pair-wise CF is applied in implicit feedback. In transfer learning, integrating matrix factorization and pair-wise CF can be the future work.
\item {\bf Online Transfer Learning in CF.} There are little work on large scale transfer learning, but it is badly desirable. In real-world, online recommendation algorithms often dominant off-line ones. Our method CBMF is a batch-updating algorithm which updates per hour, but not real-time. It would be our future work to make a real online transfer learning algorithm in CF.
\item {\bf Transfer Learning in CF with multiple matrix.} In CBMF, data from different sources are integrated into one unify matrix. Although very carefully, we can still lose or misuse the data. If we can run our algorithm fast on their original data, then we don't need to integrate.
\item {\bf Time Complexity Optimization in CF.} In \cite{Shalev-Shwartz:2008:SOI:1390156.1390273}, an interesting relationship is shown: more data can faster training speed while getting the same performance on test data. In CF there are many data that we need plenty of time to deal with. If we could leverage all of them without increasing our training time or model complexity, we could use as much data as possible.
\end{itemize} 


%%%%%%%%%%%%%%%%%%%%%%%%%%%%%%%%%%%%%%%%%%%%%%%%%%%%%%%%%%%%%%%%%%%%%%%%%
%                                                                       %
% An example of a figure. Note how the Figure Number is generated in    %
% the list of figures.                                                  %
%                                                                       %
%%%%%%%%%%%%%%%%%%%%%%%%%%%%%%%%%%%%%%%%%%%%%%%%%%%%%%%%%%%%%%%%%%%%%%%%%



%%%%%%%%%%%%%%%%%%%%%%%%%%%%%%%%%%%%%%%%%%%%%%%%%%%%%%%%%%%%%%%%%%%%%%%%%
%                                                                       %
% An example of a table. Note how the Table Number is generated in the  %
% list of tables.                                                       %
%                                                                       %
%%%%%%%%%%%%%%%%%%%%%%%%%%%%%%%%%%%%%%%%%%%%%%%%%%%%%%%%%%%%%%%%%%%%%%%%%


%%%%%%%%%%%%%%%%%%%%%%%%%%%%%%%%%%%%%%%%%%%%%%%%%%%%%%%%%%%%%%%%%%%%%%%%%
%                                                                       %
%      9) BIBLIOGRAPHY                                                  %
%                                                                       %
% This example uses bibtex to generate the required Bibliography. Refer %
% to the % the file ustthesis_test.bib for the entries of the           %
% Bibliography. Note that only the cited entries are printed.           %
%                                                                       %
% If BibTeX is not used to typeset the bibliography, replace the        %
% following line with the \begin{thebibliography} and \end{bibliography}%
% commands (the "thebibliography" environment) to process the           %
% Bibliography.                                                         %
%                                                                       %
%%%%%%%%%%%%%%%%%%%%%%%%%%%%%%%%%%%%%%%%%%%%%%%%%%%%%%%%%%%%%%%%%%%%%%%%%

%%%%%%%%%%%%%%%%%%%%%%%%%%%%%%%%%%%%%%%%%%%%%%%%%%%%%%%%%%%%%%%%%%%%%%%%%
%                                                                       %
% The recommended bibliography style is the IEEE bibliography style.    %
% "ustbib" defines the IEEE bibliography standard with the added        %
% ability of sorting the items by name of author.                       %
%                                                                       %
% If you are not using BibTeX to process your Bibliography, comment out %
% the following line.                                                   %
%                                                                       %
%%%%%%%%%%%%%%%%%%%%%%%%%%%%%%%%%%%%%%%%%%%%%%%%%%%%%%%%%%%%%%%%%%%%%%%%%
%when you want to add the references, please release the following two lines:
\bibliographystyle{plain}
\bibliography{reference}
% Please run "bibtex ustthesis_test" before the bibliography can be
% included.

%%%%%%%%%%%%%%%%%%%%%%%%%%%%%%%%%%%%%%%%%%%%%%%%%%%%%%%%%%%%%%%%%%%%%%%%%
%                                                                       %
%     10) APPENDIX (If Any)                                              %
%                                                                       %
% \appendix command marks the beginning of the APPENDIX part of the     %
% Thesis. The usual \chapter command is used for the different chapters %
% of the Appendix.                                                      %
%                                                                       %
%%%%%%%%%%%%%%%%%%%%%%%%%%%%%%%%%%%%%%%%%%%%%%%%%%%%%%%%%%%%%%%%%%%%%%%%%

%%\appendix
%%\chapter{Proof of P=NP}
%%\begin{figure}[h]
%%\caption{Illustration of Turing's Modification}
%%\end{figure}
%%\chapter{Refutation of Church-Turing Thesis}

%%%%%%%%%%%%%%%%%%%%%%%%%%%%%%%%%%%%%%%%%%%%%%%%%%%%%%%%%%%%%%%%%%%%%%%%%
%                                                                       %
%     11) BIOGRAPHY (Optional)                                          %
%                                                                       %
% \biography and \endbiography are used to define the optional          %
% Biography of the author of the Thesis.                                %
%                                                                       %
%%%%%%%%%%%%%%%%%%%%%%%%%%%%%%%%%%%%%%%%%%%%%%%%%%%%%%%%%%%%%%%%%%%%%%%%%

%\biography
% The biography of the student is ALSO optional.
%\bibliography{Publications}
%\input{Publications}
%\endbiography

\end{document}

